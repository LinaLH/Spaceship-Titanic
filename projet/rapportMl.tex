\documentclass[]{article}
\usepackage[utf8]{inputenc}
\usepackage{amsfonts} 
\usepackage{graphics}
\usepackage{listings}
\usepackage{xcolor}
\usepackage{amsmath}
\usepackage{tikz}

\definecolor{codegreen}{rgb}{0,0.6,0}
\definecolor{codegray}{rgb}{0.5,0.5,0.5}
\definecolor{codepurple}{rgb}{0.58,0,0.82}
\definecolor{backcolour}{rgb}{0.95,0.95,0.92}

\lstdefinestyle{mystyle}{
    backgroundcolor=\color{backcolour},   
    commentstyle=\color{codegreen},
    keywordstyle=\color{magenta},
    numberstyle=\tiny\color{codegray},
    stringstyle=\color{codepurple},
    basicstyle=\ttfamily\footnotesize,
    breakatwhitespace=false,         
    breaklines=true,                 
    captionpos=b,                    
    keepspaces=true,                 
    numbers=left,                    
    numbersep=5pt,                  
    showspaces=false,                
    showstringspaces=false,
    showtabs=false,                  
    tabsize=2
}

\lstset{style=mystyle}

\title{Projet Machine Learning:\\
Competition Spaceship Titanic}
\author{Valentin Hesters 20201346 \\
        Sara Firoud 20235275\\
        Lina}
\date{30 Mars 2024}



\begin{document}
\maketitle
\renewcommand*{\contentsname}{Sommaire}
\tableofcontents
   
    \section{Analyses des données}
    \par
        Les documents qui nous sont fournis sont:\\
        "sample\_submission.csv" qui est un 
        modèle de la forme que doit avoir tous fichiers de soumissions sur kaggle
        pour cette compétation.\\
        "train.csv" est la base d'entrainement des données sur lesquels nous 
        effectuerons nôtre travail d'analyse de données. Elle contient les 
        valeurs de Transported pour les deux tiers des passagers évalués et nous
        permettras donc d'entrainer nôtre modèle.\\
        Enfin, "test.csv" est le fichier sur lesquel nous devons prédire les 
        valeurs de Transported des passagers du tier restant. Elle nous serviras 
        à avoir le score final, via la soumission sur kaggle, des modèles testés.\\
        \par
        

\end{document}